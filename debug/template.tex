\documentclass[letterpaper,12pt]{article}
\usepackage[letterpaper,margin=0.75in]{geometry}
\usepackage{xcolor}
\usepackage{color}

\usepackage[english]{babel}
\usepackage[utf8x]{inputenc}
\usepackage{fullpage}
\usepackage{titlesec}
\usepackage{tcolorbox}

\newtcolorbox{minimal}{
	sharp corners,
	colback=white,
	colframe=black,
	notitle,
	before skip=1.5em,
	after skip=1.5em,
}

\usepackage{mathtools}

\usepackage{listings}

\definecolor{mygreen}{rgb}{0,0.6,0}
\definecolor{mygray}{rgb}{0.5,0.5,0.5}
\definecolor{mymauve}{rgb}{0.58,0,0.82}

\lstset{
	backgroundcolor=\color{white},
	basicstyle=\footnotesize,
	breaklines=true,
	captionpos=b,
	commentstyle=\color{mygreen},
	keywordstyle=\color{blue},
	stringstyle=\color{mymauve},
	showstringspaces=false,
}

\titleformat{\section}[block]{\color{blue}\Large\bfseries\filcenter}{}{1em}{}
\begin{document}
\section{Sample}

\begin{itemize}
	\item{Inline equations}
	\begin{itemize}
		\item[a.]{internal}
	\end{itemize}
	\item{Equation groups}
	\item{And code}
\end{itemize}

Equations within sentences work: \( x = (x + 1)^2 \) just like that

Equation groups as well

\{ 1 2 3 4 \}

\begin{minimal}
\begin{align}
P(E) = 12 \nonumber \\
\nonumber \\
{9 \choose 3} \nonumber \\
\nonumber \\
g(x) = \frac{{9 \choose 3}}{\frac{7}{2}} \nonumber
\end{align}
\end{minimal}


Some sample Python

\begin{minimal}
\begin{lstlisting}[language=python]
name = "Steve"
age = 12

\end{lstlisting}
\end{minimal}

Some text here

\begin{minimal}
\begin{lstlisting}[language=python]
print(name)
print(age)
\end{lstlisting}
\end{minimal}
\pagebreak
\begin{center}
\textsc{\large Output of code segment(s)}\\
\end{center}
\begin{minimal}
Steve\\
12
\end{minimal}
Some sample Java

\begin{minimal}
\begin{lstlisting}[language=java]
public class Test {
	
	public static void main(String[] args) {
		System.out.println("Hello, World");
	}
}
\end{lstlisting}
\end{minimal}

\begin{center}
\begin{tabular}{ |c|c|c|c| }
\hline
\multicolumn{4}{|c|}{Title} \\
\hline
	cell1 & cell2 & cell3 & cell4 \\
	\hline
	\hline
	cell5 & cell6 & cell7 & cell8 \\
	cell9 & cell10 & cell11 & cell12 \\
\hline
\end{tabular}
\end{center}
\end{document}
